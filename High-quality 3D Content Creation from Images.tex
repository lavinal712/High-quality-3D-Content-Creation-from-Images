\documentclass[cn,hazy,blue,14pt,screen]{elegantnote}
\title{High-quality 3D Content Creation from Images}

\author{Austin}
\institute{USTC}

\date{}

\usepackage{array}
\usepackage{amssymb}

\begin{document}
	
\maketitle

\tableofcontents
	
\section{Why 3D content creation from images(为什么我们要从图片生成三维内容)}

\subsection{What are 3D contents(什么是三维内容)}

3D contents are combinations of shape and appearance models that canbe rendered into 2D images from different viewpoints.

3D 内容是形状和外观组合模型,可以从不同视点渲染为 2D 图像。

Shape and appearance, where appearance consists of material and lighting, can create RGB images through rendering.

形状和外观(其中外观由材质和照明组成)可以通过渲染创建 RGB 图像。

\subsection{Rendering and inverse rendering(渲染与反渲染)}

Creating a 3D content requires strong expertise and is time-consuming.

创建 3D 内容需要强大的专业知识,而且耗时。

Images are cheap to acquire, and this leads to inverse rendering.

图片的获取成本很低,使得我们关注反渲染技术。

计算机图形学:3D contents $\Rightarrow$ RGB images

Rendering $--$ efficient, high-quality.

渲染高效,并且生成的图像质量很高。

计算机视觉:Images $\Rightarrow$ 3D contents

Inverse Rendering $--$ change lighting, material, combination of contents etc. $\Rightarrow$ new images.

反渲染能够改变原有的打光、材质、内容的组合等,生成新的图像。这些图像可能是三维内容在不同的打光环境下生成的,也可能是改变了三维内容的材质,或者是对内容进行重新组合后生成的。

\section*{Qustion 问答环节}

三维重建算反渲染吗?

三维重建是反渲染的一种,重点在重建三维内容的 Shape 方面,关注于打造高质量的三维几何模型,与 NeRF 存在不同。目前三维重建也在向 Appearance 方面进军,对内容的材质和照明也有关注。

\subsection{Three key factors in inverse rendering(反渲染中的三个重要因素)}

\begin{itemize}
\item What shape representations to use?
\item What appearance representations to use?
\item What rendering operator to use?
\end{itemize}

\subsubsection{Shape representations}

\begin{itemize}
\item Mesh 网格
\item Point cloud 点云
\item Occupancy field 占领场
\item Signed distance field(SDF) 有向距离场
\end{itemize}

\subsubsection{Appearance representations}

\begin{itemize}
\item Material texture map \& environment lighting
\item Radiance field(surface light filed\cite{wood2000surface})
\end{itemize}

第一种方法使用 unwrap 展开,在每个点存储材料。光线由 360 image 全景图生成。这是理想的外观表征方式,但是实际情况中,会变得难解和复杂。

第二种使用辐射场(或者表明光线场),打包材料和光线,使得外观表征变得简单。但是它难编辑,并且难以表示内容在新环境下的情况。

$(x,y,z,\theta,\phi)\rightarrow rgb$

\subsubsection{Rendering operators}

\begin{itemize}
\item Depend on the specific shape and appearance representations(图像的渲染取决于三维内容特定的形状和外观表征)
\item Need to be fully differentiable(ideal) (渲染在理想情况下是完全可区分的)
\end{itemize}

\section*{Qustion 问答环节}

三维重建和 NeRF 哪个效果比较好?

三维重建的效果比较好。三维重建的目标就是建立高质量的三维几何模型,而 NeRF 所建立的模型表面较为 noisy。

\section{Why NeRf is a big thing(为什么 NeRF 很重要)}

NeRF can generate highly photo-realastic novel views.

NeRF 可以生成高度真实的新视图。

\begin{example}
28 cuptured photos $\rightarrow$ NeRF(3D content) $\rightarrow$ render at unseen viewpoints.
\end{example}

\subsection{Break NeRF into 3 components(NeRF 的三个组成部分)}

\begin{itemize}
\item Shape\\
Soft opacity field(fog) 软不透明度场\\
$(x,y,z)\rightarrow\sigma$
\item Appearance\\
Radiance field(surface light filed)\\
$(x,y,z,\theta,\phi)\rightarrow rgb$
\item Rendering\\
$(x,y,z,\theta,\phi)\rightarrow F_\Theta\rightarrow(RGB\sigma)$
\end{itemize}

\subsection{Take-home message from NeRF(从 NeRF 中能了解什么)}

\subsubsection{Soft shape}

\begin{itemize}
\item initialized as nothingness(初始化为空)
\item grow as needed(依据需要而变)
\item compare with hard geometry, e.g., DVR\cite{niemeyer2020differentiable}, IDR\cite{yariv2020multiview}.
\begin{itemize}
\item $\checkmark$ Not require object segmentation masks: genus issue.
\item $\checkmark$ No boundary discountinuity: easy for differentiable rending.
\item $\times$ Rendering is more expensive; editing is hard.
\end{itemize}
\item without neural networks, but soft shape: fast.\cite{fridovich2022plenoxels}\cite{sun2022direct}
\end{itemize}

\subsubsection{Fourier features fixes the spectal bias of MLPs(傅里叶特征法修复了 MLP 的隐含偏差)}

\[\gamma(\bold{v})=[\cdots,\cos(2\pi\sigma^{j/m}\bold{v}),\sin(2\pi\sigma^{j/m}\bold{v}),\cdots]^{\mathrm{T}}\]

\subsubsection{Neural fields for inverse rendering(用于反渲染的神经场)}

\begin{itemize}
\item Compact: each NeRF(3D content) is only about 10 MB.
\begin{itemize}
\item small
\item free to transport
\end{itemize}
\item Continuous: scenes are not discretized.
\begin{itemize}
\item high-quality
\end{itemize}
\item Flexible and easy for optimization: implicit regularizationfrom neural networks.
\end{itemize}

\subsection{Unbounded scene and anti-aliasing(无界场景与抗锯齿)}

\subsubsection{Five capture scenarios(五种拍摄场景)}

\begin{itemize}
\item 360 Inward-facing (no background)
\item 360 Outward-facing
\item Foreward-facing
\item Unconstrained 360 Outward-facing (irregular)
\item 360 Inward + Outward-facing
\end{itemize}

\subsubsection{NeRF++}

NeRF's resolution for 360 capture of unbounded scene: local or overall?

NeRF++: separate foreground and background; semmetry between 360 inward-facing and 360 outward-facing captures.\cite{zhang2020nerf++}

\begin{itemize}
\item Map unbounded space outside a sphere into a unit cube.
\item Farther-away points get more "squeezed" after mapping.
\end{itemize}

\subsubsection{Conclusion}

\begin{itemize}
\item NeRF faces resolution for 360 capture of unbounded scene.
\item NeRF++ separates foreground and background modelling, as NeRF is compositional.
\item NeRF++ uses inverted sphere parametrization to establish semmetry between 360 inward-facing and 360 outward-facing captures.
\end{itemize}

\subsubsection{NeRF's anti-aliasing issue}

Nyquist-Shannon sampling theorem(fs$>$2B)

key: low pass filter

MipNeRF: low-pass filter the Fourier features(Fourier features are vector functions of the spatial location).

\[\bold{v}=(x,y,z)\].

\begin{itemize}
\item NeRF suffers from anti-aliasing issue.
\item Images are discrete signals.
MipNeRF introduces low-pass filtering over the Fourier features, where the filter size is controlled by the cone size.\cite{barron2021mip}
\end{itemize}

\section{Editable 3D contents(可编辑三维内容)}

\subsection{Relighting, material re-touching(重打光与材料编辑)}

\subsubsection{What NeRF can and can't do}

\begin{itemize}
\item NeRF is great 3D content in term of synthesizing highly photo realistic novel views.
\item Future application:
\begin{itemize}
\item Put mutiple NeRF in the same environment
\item Retouch appearance
\item Create artistic 3D contents 
\item Interactive editings
\item Physical simulation
\item Object insertion
\item Animations
\end{itemize}
\end{itemize}

Mesh and material textures remain dominate 3D content format today.

Input: photomatric images $\rightarrow$ Output: 3D assets

But mesh and material textures are hard to optimize directly.
\begin{itemize}
\item Fixed topology, fixed resolution
\item Hard to avoid self-intersections
\end{itemize}

Leverage the insights from NeRF to ease optimization?
\begin{itemize}
\item During optimization time, use neural field for shape and appearance representations.
\begin{itemize}
\item Use soft shape for quickly instantiating geometry from nothingness.
\item Use hard shape for physics-based rendering.
\end{itemize}
\item During test-time, use mesh and material textures.
\end{itemize}

\subsubsection{Stage of NeRF}

\begin{itemize}
\item Optimize a thick surface via volumetric rendering.
\item Refine a surface and solve for material \& lighting.
\begin{itemize}
\item Edge-aware surface rendering
\begin{itemize}
\item Shading at edge pixels(visiblity discontinuities) contain shape cues.
\item Sub-optimal shape reconstruction without accounting for such edges.
\item Split an edge pixel into havles, and blend color of both halves.
\end{itemize}
\end{itemize}
\end{itemize}

\subsection{Stylized appearance(风格化的外观)}

\subsubsection{Artistic 3D content creation through stylization(通过风格化进行艺术 3D 内容创作)}

Style loss $\underleftarrow{\mathrm{render\ \&\ compare}}$ 3D content like NeRF $\underrightarrow{\mathrm{render\ \&\ compare}}$ L1/L2 loss

Style loss measures mismatch in feature sets.

\subsubsection{Nearest Neighbor Feature Matching(NNFM) style loss}

\[\ell_{nnfm}(\bold{F}_{render},\bold{F}_{style})=\dfrac{1}{N}\sum_{i,j}\min_{i',j'}D(\bold{F}_{render}(i,j),\bold{F}_{style}(i',j'))\]
\[D(\bold{v}_1,\bold{v}_2)=1-\bold{v}_1^{\mathrm{T}}\bold{v}_2\sqrt{\bold{v}_1^{\mathrm{T}}\bold{v}_1\bold{v}_2^{\mathrm{T}}\bold{v}_2}\]

\subsection{Other things(其他内容)}

ARF project page: \href{https://www.cs.cornell.edu/projects/arf}{https://www.cs.cornell.edu/projects/arf}

\subsubsection{Paper and resourses}

\begin{itemize}
\item Mildenhall et al. ECCV 2020. NeRF: Representing Scenes as Neural Radiance Fields for View Synthesis.
\item Zhang et al. ArXiv 2020. Nerf++: Analyzing and Improving Neural Radiance Fields.
\item Barron et al. CVPR 2021. Mip-NeRF: A Multiscale Reprentation for Anti-aliasing Neural Radiance Fields.
\item Zhang et al. CVPR 2022. IRON: Inverse Rendering by Optimizing Neural SDFs and Materials from Photometric Images.
\item Zhang et al. ECCV 2022. ARF: Artistic Radiance Fields.
\item Awesome NeRF: \href{http://github.com/yenchelin/awesome-NeRF}{http://github.com/yenchelin/awesome-NeRF}.
\end{itemize}

\bibliographystyle{IEEEtran}
\bibliography{reference}

\end{document}